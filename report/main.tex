\documentclass[11pt,a4paper]{article}

% 使用自定义样式包 (支持中文,需要 XeLaTeX 编译)
% 编译命令: xelatex main.tex
\usepackage{reportstyle}

% 设置页眉内容
\setreportheader{Example PDF}{\today}

% 标题和作者设置
\title{Example PDF}
\author{Author}
\date{\today}

\begin{document}

% 自定义标题页眉(不使用\maketitle)
\maketitle
\thispagestyle{fancy}

% 生成目录
\tableofcontents
\newpage

\section{Images and Tables}

\subsection{LaTeX Table with Caption}

At vero eos et accusam et justo duo dolores et ea rebum. Stet clita kasd gubergren, no sea takimata sanctus est Lorem ipsum dolor sit amet. Lorem ipsum dolor sit amet, consetetur sadipscing elitr.

\begin{table}[H]
\centering
\caption{Nam liber tempor cum soluta nobis eleifend option congue nihil imperdiet doming id quod maxim placerat facer possim assum. Lorem ipsum dolor sit amet, consectetur adipiscing elit, sed diam nonummy nibh euismod tincidunt ut laoreet dolore magna aliquam erat volutpat.}
\label{tab:test_results}
\begin{tabular}{@{}cccccccc@{}}
\toprule
\textbf{Test Nr.} & \textbf{Position} & \textbf{Radius} & \textbf{Rot} & \textbf{Grün} & \textbf{Blau} & \textbf{beste Fitness} & \textbf{Abweichung} \\
\midrule
1 & 20\% & 20\% & 20\% & 20\% & 20\% & 7,5219 & 0,9115 \\
2 & 0\% & 25\% & 25\% & 25\% & 25\% & 8,0566 & 1,4462 \\
3 & 0\% & 0\% & 33\% & 33\% & 33\% & 8,7402 & 2,1298 \\
4 & 50\% & 20\% & 10\% & 10\% & 10\% & 6,6104 & 0,0000 \\
5 & 70\% & 0\% & 10\% & 10\% & 10\% & 7,0696 & 0,4592 \\
6 & 20\% & 50\% & 10\% & 10\% & 10\% & 7,0034 & 0,3930 \\
\bottomrule
\end{tabular}
\end{table}

At vero eos et accusam et justo duo dolores et ea rebum. Stet clita kasd gubergren, no sea takimata sanctus est Lorem ipsum dolor sit amet. Lorem ipsum dolor sit amet, consetetur sadipscing elitr.

\subsection{Image with Caption}

\begin{figure}[H]
\centering
    % 使用智能图片插入命令
    \smartimage[0.8\textwidth]{example-image.png}{示例图片}{请将您的图片文件命名为 example-image.png 或修改图片路径}
\caption{Nam liber tempor cum soluta nobis eleifend option congue nihil imperdiet doming id quod maxim placerat facer possim assum. Lorem ipsum dolor sit amet, consectetur adipiscing elit, sed diam nonummy nibh euismod tincidunt ut laoreet dolore magna aliquam erat volutpat.}
\label{fig:example}
\end{figure}

\section{样式包功能演示}

\subsection{文本样式}

这是普通文本。\emphtext{这是强调文本}。\highlight{这是高亮文本}。这是\code{代码文本}。

\subsection{信息框演示}

\begin{infobox}
    \textbf{信息:} 这是一个信息框,用于显示重要信息。
\end{infobox}

\begin{warningbox}
    \textbf{警告:} 这是一个警告框,用于显示需要注意的内容。
\end{warningbox}

\begin{successbox}
    \textbf{成功:} 这是一个成功框,用于显示成功完成的操作。
\end{successbox}

\subsection{自定义颜色}

使用了自定义颜色方案:
\begin{itemize}
    \item \textcolor{primarycolor}{主要颜色 (Primary Color)}
    \item \textcolor{secondarycolor}{次要颜色 (Secondary Color)}
    \item \textcolor{accentcolor}{强调颜色 (Accent Color)}
\end{itemize}

\begin{note}
    \textbf{侧边栏:} 这是一个侧边栏,用于显示重要信息~\cite{key}。
\end{note}

\begin{note}[accentcolor]
    \textbf{橙色边框:} 这是使用强调颜色的note。
\end{note}

\begin{note}[secondarycolor]
    \textbf{蓝色边框:} 这是使用次要颜色的note。
\end{note}

\begin{note}[red]
    \textbf{红色边框:} 这是使用红色的note。
\end{note}

\begin{note}[blue!80!black]
    \textbf{深蓝色边框:} 这是使用深蓝色的note。
\end{note}

\bibliographystyle{plain}
\bibliography{ref}

\end{document}
